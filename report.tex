\documentclass[12pt, conference, final, a4paper, onecolumn, compsoc]{IEEEtran}
% Font size onecolumn or twocolumn Use draft for notes or final for no spacing

% Includegraphics
\usepackage{graphicx}
% Bibliography
\usepackage{natbib}
% Figure positions
\usepackage{float}
% Wrapping URLs
\usepackage[hyphens]{url}

\begin{document}


\title{Using AI to Detect Malware in Object Storage} \author{Author: Matthew
  Battagel, Supervisor: Theodoros Spyridopoulos} \markboth{Cardiff University -
  CM3203 - Final Report}{}
\maketitle{}

\subsection*{Acknowledgments - }
% Remove section from TOC
\addtocontents{toc}{\protect\setcounter{tocdepth}{-1}}
\paragraph{}
I would like to extend my sincere gratitude to my supervisor Theo, my colleague
Harry, friends, family, and Lois for their unwavering support and encouragement
during my project. Their combined expertise and guidance provided were critical
in the shaping and execution of the project. I am truly grateful to all of them
for their contributions.

\bigskip

\begin{abstract}
    Lorem Ipsum
\end{abstract}

\pagebreak

    % Problem and background Understanding of the problem and the aims and
    % objectives of the project Awareness of the background of the problem

    % Detailed analysis of the problem, suitability of approach towards solving
    % the problem Solution to the problem Approach and design Solution,
    % implementation Use of and justification for appropriate tools/methods

    % Evaluation Testing and validation Critical appraisal of results

    % Achievement of agreed overall deliverables given in the initial plan for
    % the final report (or a justified modification of these) Communication and
    % project management skills Written communication skills Project planning,
    % control and reflection Interaction and work with the supervisor


    % Contents
    \tableofcontents{}


    \section{Introduction}
    \subsection*{Overview}
    \paragraph{} % Data growth and object storage
    The exponential growth of data generation has made data storage an
    increasingly important aspect for both individuals and organizations alike.
    Object storage has emerged as a promising solution due to its ability to
    store vast amounts of unstructured data in a cost-effective and scalable
    manner. Unlike traditional storage techniques, object storage stores data as
    objects with related metadata and unique identifiers, allowing for efficient
    and cheap storage within buckets.

    % !! REWORD LAST BIT

    \paragraph{} % Market and competition
    One of the most widely used object storage platforms is Amazon S3, which
    provides a highly scalable and reliable solution for storing data. However,
    an open-source alternative called MinIO has emerged as a promising
    contender, providing similar features to Amazon S3 while giving customers
    greater control over their data. MinIO is written in Go and is available for
    free under the Apache License 3.0 or, for commercial and enterprise
    purposes, at a reduced cost compared to Amazon S3. \citep{minio-pricing}.
    MinIO offers a wide range of features, including high performance, data
    replication, encryption and erasure coding \citep{minio}. Most importantly,
    MinIO is designed to scale out horizontally to ensure that it can handle the
    demands of large-scale applications.

    \paragraph{} %Scalability
    Scalability is made simple by allowing multiple types of hardware platforms
    to work together in separate nodes each with their own compute and storage.
    This is extremely attractive for customers who want to utilises their
    existing hardware without being tied down to a specific provider. This also
    applies for customers looking to migrate their data from Amazon S3 to
    cheaper solution without compromising on the high performance, reliability
    and scalability of the S3 platform.

    \paragraph{} %Downsides
    While MinIO is a great alternative to Amazon S3, it does not offer any form
    of malware detection integration. This could put customers off from choosing
    MinIO as a viable platform to migrate to from Amazon S3 or leave existing
    users data vulnerable to malware attacks. This project aims to address this
    issue by integrating a malware detection system into MinIO. An important
    goal for the is to negatively impact the scalability or performance as
    little as possible so that MinIO is still an effective alternative to Amazon S3.

    \subsection*{Motivation}
    % Why am I trying to add malware detection for object storage?
    % Can I say about HPE? Or shall I say its for the good of the world
    \paragraph{}
    Due to the high amount of unstructured data expected to be both written and
    read to the object store, there are increased risk of encountering malicious
    files. Therefore malware detection within object storage is crucial in
    modern cloud storage scenarios. Most popular off-the-shelf object storage
    platforms, such as AWS, already have integrated third-party antivirus
    software, such as ClamAV and Sophos \citep{amazon-md}, to mitigate security
    risks. MinIO on the other hand is vulnerable to malware attacks as it
    currently does not have any native antivirus integration. This forces
    customers who require complete virus protection to either not use MinIO or
    to use potentially costly third-party software. As antivirus scanning is
    inherently resource intensive, if the software is integrated incorrectly, it
    could reduce the ability for the storage solution to scale horizontally
    which negates one of the major benefits of object storage. The purpose of
    this project is to implement malware detection within MinIO while being
    mindful to not impact the scalability or performance of the platform.

    \subsection*{Project Aims}
    % Implement an anti virus / malware detection system within an object
    % storage platform
    \paragraph{}
    This project aims to supply an end-to-end solution for detecting malware
    within the MinIO object storage platform with three main requirements. The
    solution should be able to detect the latest uploads to the object store and
    scan them. It should be able to perform this function without significantly
    impacting the performance of the object store. The solution should also
    scale alongside MinIO to ensure that it does not bottleneck the object store
    at high loads. These main three aims can be quantified so that an accurate
    evaluation of the solution can be made at the end of the project.

    \begin{itemize}
      \item Detection of malware within the object store should match 100\%
      of the malware detection rate as the stand-alone antivirus.
      \item Performance of the solution to be within 10\% of the performance of
      the stand-alone object store, MinIO.
      \item Retain the previous metric while both increasing the available
      resources and changing the platform.
    \end{itemize}

    \subsection*{Important outcomes / targets}
    % Overview of project Aims Outcomes
    \paragraph{}


    \section{Background}

    % background should be more previous research material etc. Include
    % competition and potential software to use e.g. ClamAV, MinIO.
    % Include the first things you find when googling the same topic as diss.

    % Where to put background info on object storage and malware? Not sure
    %

    % Insert background material
    %


    \subsection*{Amazon S3 Malware Detection} % How does amazon do it?
    \paragraph{}
    As MinIO's largest competitor, this project draws a lot of inspiration from
    Amazon S3s integrated malware detection blog page \citep{amazon-md}. The
    blog explains Amazons current approach for managing malware detection within
    their service. Amazon S3 uses a combination of ClamAV and Sophos as their
    third-party scanning engines due to their out-of-the-box nature. Amazon then
    gives you the option to use either of these engines or both. The blog goes
    on to describe the three main interaction mechanisms that Amazon S3 uses to
    flag files for scanning. Firstly, an API endpoint would be provided to
    handle all uploads. This forms a queue of uploads which are then
    scanned before entering the bucket. Next, event-driven scanning is used keep
    track of all regular file uploads. The antivirus will then scan each file
    after they have been written to the bucket. Finally, retro-driven scanning
    is used to scan all existing files within the bucket. The user then has the
    flexibility to define what types of files should be scanned including
    defining time windows. This blog has given some useful methodologies of how
    to keeping track of both incoming and previously scanned files. Creating a
    system that can match these methods is important for offering a matching
    level of scalability and security within MinIO.

    \subsection*{} % How does signature detection work?
    \paragraph{}

    \subsection*{} % Best ways of implementing AV into a micro-service?
    \paragraph{}

    \subsection*{} % How does ClamAV work?
    \paragraph{}
    % Malware Artificial Intelligence Object Storage Context

    \section{Specification and Design}

    % System requirements specification ***

    % Optimal architecture Packaging with K8s Make sure to match with
    % specification

    \subsection*{Specification}
    \paragraph{}

    \subsection*{Architecture}
    \subsubsection*{Design 1}
    \paragraph*{}

    \paragraph{}
    \subsubsection*{Design 3}

    \subsubsection*{Design 2}
    \paragraph{}

    \subsubsection*{Design 4}
    \paragraph{}

    \subsubsection*{Optimal Design}
    \paragraph{}

    \section{Implementation}
    \subsection*{Event Bus}
    % Kafka

    \subsection*{Packaging}

    \section{Results and Evaluation}
    % Evaluate against specification
    \subsection*{}

    \section{Future Work}
    \subsection*{}

    \section{Conclusions}
    \subsection*{}

    \section{Reflection on Learning}
    \subsection*{}

    \section{Appendix}
    \bibliographystyle{cardiff} \bibliography{references}

  \end{document}

  % TEMPLATED
  % \begin{figure}[!ht]
  %   \centering \includegraphics[scale=.55]{assets/extracred}
  %   \caption{Bouncing and pitching motion as a function of time}
  %   \label{fig:bounceAndPitch}
  % \end{figure}

  % \begin{table}[!ht]
  %   \begin{center}
  %     \caption{Calculated values}
  %     \label{tab:calculated}
  %     \begin{tabular}{|c|c|}
  %       \hline
  %     \end{tabular}
  %   \end{center}
  % \end{table}

  % Appendix \onecolumn \textwidth=456pt \paperwidth=577pt \hoffset=-30pt
  % \newpage
  % \newpage
  % \clearpage


  % \pagestyle{headings}
  % \section{Appendix A}
  % \centering Heading \footnotesize

  % \normalsize

  % \csvautolongtable{assets/vibProj.csv}
    % \subsection*{Malware}
    % \paragraph{}
    % Malware is a type of software designed to harm or exploit computer systems,
    % networks, and users. It includes a wide range of harmful programs, such as
    % viruses, worms, Trojans, ransom-ware, spyware, and adware. Malware can be
    % used to steal sensitive information, gain unauthorized access to systems,
    % damage or destroy data, and perform other malicious activities. Malware can
    % be distributed through various channels, such as email attachments,
    % malicious web-sites, software downloads, and infected removable media. It is
    % a serious threat to computer security and can cause significant damage to
    % storage devices if left untreated.
    % % !! And has the potential to be laying dormant inside peoples computers
%
    % \paragraph{}
    % Malicious files can be particularly risky for storage devices in cloud
    % environments due to their shared nature. In a cloud storage environment,
    % multiple users and applications can access and store data on the same
    % physical storage infrastructure. This means that an infected file can
    % quickly spread and infect other files and users, compromising the security
    % and integrity of the entire system. Moreover, cloud storage providers may
    % not be able to isolate and contain malware as easily as with traditional
    % storage systems. As a result, cloud storage users need to be extra vigilant
    % and take proactive measures, such as implementing antivirus software,
    % regular backups, and secure access controls, to mitigate the risks of
    % malware infections.
%
    % \subsection*{Anti-Virus}
    % % Create a connection between paragraphs here.
    % % Include Citation
    % \paragraph{}
    % In the context of modern IT solutions, ensuring robust security measures is
    % crucial to guarantee reliable operations and safeguard data against threats.
    % Companies can face severe penalties, up to XXXX, for failing to comply with
    % data protection regulations in the event of a breach. Therefore, it is
    % imperative that IT products offer comprehensive security features, mainly
    % including built-in antivirus capabilities, to protect against malicious
    % attacks and prevent data loss.
%
    % \paragraph{}
    % Modern Anti-Viruses mostly work using a combination of signature-based and
    % heuristic detection methods. Signature-based detection works by comparing
    % the file being scanned against a database of known malicious file
    % signatures. If a match is found, the file is flagged as malicious. This
    % method is effective at detecting known malware, but it is not effective at
    % detecting new or unknown malware. Heuristic detection, on the other hand,
    % works by analyzing the behavior of the file being scanned and comparing it
    % against a database of known malicious behaviors. This method is more
    % effective as it looks at practices used by malware, such as common imports,
    % and therefore has potential to detect previously unknown malware.
%
    % % EXPLAIN HOW MACHINE LEARNING CAN BE USED??
    % % Tie it into next paragraphs
%
    % \subsection*{Artificial Intelligence}
    % % What is AI?
    % % Why use AI?
    % % Is it better than regular detection?
    % % How does this affect our definition of malware and antivirus?
%
    % \paragraph{}
    % % SUBSEQUENTLY EXPLAIN WHAT AI IS?
%
    % \subsection*{Object Storage}
    % \paragraph{}
    % Object storage is a type of data storage architecture that manages data as
    % discrete units known as objects. Each object contains data, metadata
    % (information about the object), and a unique identifier that enables it to
    % be located and retrieved. Object storage systems are designed to handle
    % large amounts of unstructured data, such as files, images, videos, and other
    % multimedia content and have become increasingly popular in recent years with
    % a compound annual growth rate of 13.6\% \citep{object-storage-market}.
%
    % \paragraph{}
    % Unlike traditional storage architectures like file or block storage, object
    % storage does not organize data in a hierarchical directory structure or use
    % fixed-sized blocks. Instead, it allows data to be stored and accessed
    % independently of the underlying physical storage infrastructure. This makes
    % it easier to scale storage capacity and performance, as well as to implement
    % features like data replication, versioning, and encryption. Object storage
    % is often used in cloud computing environments, where it is a popular option
    % for storing and managing data in distributed systems.
%
%
    % \subsection*{MinIO}
    % \paragraph{}
    % MinIO is a high-performance, open-source, distributed object storage system
    % that is designed to be an alternative to Amazon S3. It is S3 compatiable
    % allowing for almost instantanous migration from Amazon to MinIO. MinIO is
    % also designed to
    % scale horizontally and can be deployed on a wide variety of hardware and
    % software platforms \citep{minio}. It is written in Go and is available under
    % the Apache License 2.0 \citep{minio-repo}. MinIO has several features that
    % make it an attractive choice for cloud-native applications. It is
    % fault-tolerant, with data being automatically distributed across multiple
    % drives and servers (erasure) to ensure high availability. Additionally,
    % MinIO is highly performant, with a focus on optimizing for SSDs and NVMe
    % drives \citep{minio}.
%
    % \paragraph{}
    % Currently, MinIO does not provide built-in antivirus capabilities. This
    % means that users must rely on third-party antivirus software to protect
    % their data from malware. However, this can be a challenge for users who want
    % to use MinIO in cloud environments, as antivirus software is often
    % incompatible with cloud-native applications. This is because antivirus
    % software is designed to run on a single machine and is not designed to scale
    % horizontally. As a result, antivirus software is not well-suited for
    % cloud-native applications, which are often deployed in distributed
    % environments. Moreover, antivirus software is often resource-intensive and
    % can slow down the performance of cloud-native applications if scalability is
    % not properly considered. This is particularly problematic for cloud-native
    % applications that require high performance, such as video streaming and
    % machine learning.
%
    % \paragraph{}
    % Adding an in-built antivirus capabilities to MinIO would allow users to
    % protect their data from malware without having to rely on third-party tools.
    % This would also allow users to run antivirus software in a cloud-native with
    % scalability and performance in mind. In this report we will discuss
    % different approaches that can be taken to embed an antivirus engine into
    % MinIO and evaluate their performance and scalability.
%
    % \subsection*{ClamAV}
    % % Describe inner workings!?!?!
    % % Open-source
%
    % \paragraph{}
    % Open-source antivirus software is a popular choice for cloud-native and
    % distributed applications. This is because open-source software is often
    % free, which makes it a cost-effective option for users. Additionally,
    % open-source software is often more customizable and flexible than
    % proprietary software, which makes it easier to integrate into cloud-native
    % applications. Open-source software is also often more secure than
    % proprietary software, as it is often reviewed by a large community of
    % developers and users.
%
    % \paragraph{}
    % % !!CHECK does it focus on distribution??
    % One of the most popular open-source antivirus software is ClamAV. It is
    % written in C and is available under the GNU General Public License
    % \citep{clamav-repo}. ClamAV is designed to be lightweight and fast, with a
    % focus on distributed and scalable scanning. ClamAV uses a combination of
    % signature-based and heuristic detection methods as mentioned earlier with a
    % reported accuracy of XXXXX. It uses an open-source database of known
    % malicious files and behaviors which is updated regularly by the ClamAV
    % community.
%
    % \paragraph{}
    % % !!CHECK is this what we do in the end??
    % % Can we use AI to make it better? How?
    % For prototype implementation and benchmarking of the system architecture, we
    % will use ClamAV as our antivirus engine. This can then be replaced or
    % upgraded with a more advanced antivirus engine that utilises AI.
