\documentclass[12pt, conference, final, a4paper, onecolumn, compsoc]{IEEEtran}
% Font size onecolumn or twocolumn Use draft for notes or final for no spacing

% Includegraphics
\usepackage{graphicx}
% Bibliography
\usepackage{natbib}
% Figure positions
\usepackage{float}
% Wrapping URLs
\usepackage[hyphens]{url}

\begin{document}


\title{Using AI to Detect Malware in Object Storage} \author{Author: Matthew
  Battagel, Supervisor: Theodoros Spyridopoulos} \markboth{Cardiff University -
  CM3203 - Final Report}{}
\maketitle{}

\subsection*{Acknowledgments - }
% Remove section from TOC
\addtocontents{toc}{\protect\setcounter{tocdepth}{-1}}
\paragraph{}
I would like to extend my sincere gratitude to my supervisor Theo, family,
friends, colleagues, and girlfriend Lois for their unwavering support and
encouragement during my research project. The guidance and expertise provided by
my supervisor and colleagues were critical in the shaping and success of the
project. The support and constructive feedback from all parties helped me to
refine my work notably Lois's constant motivation and dedication were a great
source of strength throughout the project. I am truly grateful to all of them
for their contributions.

\bigskip

\begin{abstract}
    Lorem Ipsum
\end{abstract}

\pagebreak

    % Problem and background Understanding of the problem and the aims and
    % objectives of the project Awareness of the background of the problem

    % Detailed analysis of the problem, suitability of approach towards solving
    % the problem Solution to the problem Approach and design Solution,
    % implementation Use of and justification for appropriate tools/methods

    % Evaluation Testing and validation Critical appraisal of results

    % Achievement of agreed overall deliverables given in the initial plan for
    % the final report (or a justified modification of these) Communication and
    % project management skills Written communication skills Project planning,
    % control and reflection Interaction and work with the supervisor


    % Contents
    \tableofcontents


    \section{Introduction}
    \subsection*{Overview}
    \paragraph{}

    \subsection*{Motivation} % Yes
    % Why am I trying to add malware detection for object storage?
    % Can I say about HPE? Or shall I say its for the good of the world
    \paragraph{}

    \subsection*{Project Aims}
    % Implement an anti virus / malware detection system within an object
    % storage platform
    \paragraph{}

    \subsection*{Important outcomes / targets}
    % Overview of project Aims Outcomes
    \paragraph{}


    \section{Background}

    % background should be more previous research material etc. Include
    % competition and potential software to use e.g. ClamAV, MinIO.
    % Include the first things you find when googling the same topic as diss.

    % Where to put background info on object storage and malware? Not sure

    \subsection*{Malware}
    \paragraph{}
    Malware is a type of software designed to harm or exploit computer systems,
    networks, and users. It includes a wide range of harmful programs, such as
    viruses, worms, Trojans, ransom-ware, spyware, and adware. Malware can be
    used to steal sensitive information, gain unauthorized access to systems,
    damage or destroy data, and perform other malicious activities. Malware can
    be distributed through various channels, such as email attachments,
    malicious web-sites, software downloads, and infected removable media. It is
    a serious threat to computer security and can cause significant damage to
    storage devices if left untreated.
    % !! And has the potential to be laying dormant inside peoples computers

    \paragraph{}
    Malicious files can be particularly risky for storage devices in cloud
    environments due to their shared nature. In a cloud storage environment,
    multiple users and applications can access and store data on the same
    physical storage infrastructure. This means that an infected file can
    quickly spread and infect other files and users, compromising the security
    and integrity of the entire system. Moreover, cloud storage providers may
    not be able to isolate and contain malware as easily as with traditional
    storage systems. As a result, cloud storage users need to be extra vigilant
    and take proactive measures, such as implementing antivirus software,
    regular backups, and secure access controls, to mitigate the risks of
    malware infections.

    \subsection*{Anti-Virus}
    % Create a connection between paragraphs here.
    % Include Citation
    \paragraph{}
    In the context of modern IT solutions, ensuring robust security measures is
    crucial to guarantee reliable operations and safeguard data against threats.
    Companies can face severe penalties, up to XXXX, for failing to comply with
    data protection regulations in the event of a breach. Therefore, it is
    imperative that IT products offer comprehensive security features, mainly
    including built-in antivirus capabilities, to protect against malicious
    attacks and prevent data loss.

    \paragraph{}
    Modern Anti-Viruses mostly work using a combination of signature-based and
    heuristic detection methods. Signature-based detection works by comparing
    the file being scanned against a database of known malicious file
    signatures. If a match is found, the file is flagged as malicious. This
    method is effective at detecting known malware, but it is not effective at
    detecting new or unknown malware. Heuristic detection, on the other hand,
    works by analyzing the behavior of the file being scanned and comparing it
    against a database of known malicious behaviors. This method is more
    effective as it looks at practices used by malware, such as common imports,
    and therefore has potential to detect previously unknown malware.

    % EXPLAIN HOW MACHINE LEARNING CAN BE USED??
    % Tie it into next paragraphs

    \subsection*{Artificial Intelligence}
    % What is AI?
    % Why use AI?
    % Is it better than regular detection?
    % How does this affect our definition of malware and antivirus?

    \paragraph{}
    % SUBSEQUENTLY EXPLAIN WHAT AI IS?

    \subsection*{Object Storage}
    \paragraph{}
    Object storage is a type of data storage architecture that manages data as
    discrete units known as objects. Each object contains data, metadata
    (information about the object), and a unique identifier that enables it to
    be located and retrieved. Object storage systems are designed to handle
    large amounts of unstructured data, such as files, images, videos, and other
    multimedia content and have become increasingly popular in recent years with
    a compound annual growth rate of 13.6\% \citep{object-storage-market}.

    \paragraph{}
    Unlike traditional storage architectures like file or block storage, object
    storage does not organize data in a hierarchical directory structure or use
    fixed-sized blocks. Instead, it allows data to be stored and accessed
    independently of the underlying physical storage infrastructure. This makes
    it easier to scale storage capacity and performance, as well as to implement
    features like data replication, versioning, and encryption. Object storage
    is often used in cloud computing environments, where it is a popular option
    for storing and managing data in distributed systems.

    \subsection*{MinIO}
    \paragraph{}
    MinIO is a high-performance, open-source, distributed object storage system
    that is designed to be cloud-native. It provides Amazon S3-compatible API
    for developers to build cloud-native applications. MinIO is designed to
    scale horizontally and can be deployed on a wide variety of hardware and
    software platforms \citep{minio}. It is written in Go and is available under
    the Apache License 2.0 \citep{minio-repo}. MinIO has several features that
    make it an attractive choice for cloud-native applications. It is
    fault-tolerant, with data being automatically distributed across multiple
    drives and servers (erasure) to ensure high availability. Additionally,
    MinIO is highly performant, with a focus on optimizing for SSDs and NVMe
    drives \citep{minio}.

    \paragraph{}
    Currently, MinIO does not provide built-in antivirus capabilities. This
    means that users must rely on third-party antivirus software to protect
    their data from malware. However, this can be a challenge for users who want
    to use MinIO in cloud environments, as antivirus software is often
    incompatible with cloud-native applications. This is because antivirus
    software is designed to run on a single machine and is not designed to scale
    horizontally. As a result, antivirus software is not well-suited for
    cloud-native applications, which are often deployed in distributed
    environments. Moreover, antivirus software is often resource-intensive and
    can slow down the performance of cloud-native applications if scalability is
    not properly considered. This is particularly problematic for cloud-native
    applications that require high performance, such as video streaming and
    machine learning.

    \paragraph{}
    Adding an in-built antivirus capabilities to MinIO would allow users to
    protect their data from malware without having to rely on third-party tools.
    This would also allow users to run antivirus software in a cloud-native with
    scalability and performance in mind. In this report we will discuss
    different approaches that can be taken to embed an antivirus engine into
    MinIO and evaluate their performance and scalability.

    \subsection*{ClamAV}
    % Describe inner workings!?!?!
    % Open-source

    \paragraph{}
    Open-source antivirus software is a popular choice for cloud-native and
    distributed applications. This is because open-source software is often
    free, which makes it a cost-effective option for users. Additionally,
    open-source software is often more customizable and flexible than
    proprietary software, which makes it easier to integrate into cloud-native
    applications. Open-source software is also often more secure than
    proprietary software, as it is often reviewed by a large community of
    developers and users.

    \paragraph{}
    % !!CHECK does it focus on distribution??
    One of the most popular open-source antivirus software is ClamAV. It is
    written in C and is available under the GNU General Public License
    \citep{clamav-repo}. ClamAV is designed to be lightweight and fast, with a
    focus on distributed and scalable scanning. ClamAV uses a combination of
    signature-based and heuristic detection methods as mentioned earlier with a
    reported accuracy of XXXXX. It uses an open-source database of known
    malicious files and behaviors which is updated regularly by the ClamAV
    community.

    \paragraph{}
    % !!CHECK is this what we do in the end??
    % Can we use AI to make it better? How?
    For prototype implementation and benchmarking of the system architecture, we
    will use ClamAV as our antivirus engine. This can then be replaced or
    upgraded with a more advanced antivirus engine that utilises AI.

    \subsection*{}
    % Malware Artificial Intelligence Object Storage Context

    \section{Specification and Design}

    % System requirements specification ***

    % Optimal architecture Packaging with K8s Make sure to match with
    % specification

    \subsection*{Specification}
    \paragraph{}

    \subsection*{Architecture}
    \subsubsection*{Design 1}
    \paragraph*{}

    \subsubsection*{Design 2}
    \paragraph{}

    \subsubsection*{Design 3}
    \paragraph{}

    \subsubsection*{Design 4}
    \paragraph{}

    \subsubsection*{Optimal Design}
    \paragraph{}

    \subsection*{Packaging}


    \section{Implementation}
    \subsection*{}
    \subsection*{Packaging}

    \section{Results and Evaluation}
    % Evaluate against specification
    \subsection*{}

    \section{Future Work}
    \subsection*{}

    \section{Conclusions}
    \subsection*{}

    \section{Reflection on Learning}
    \subsection*{}

    \section{Appendix}
    \bibliographystyle{cardiff} \bibliography{references}

  \end{document}

  % TEMPLATED
  % \begin{figure}[!ht]
  %   \centering \includegraphics[scale=.55]{assets/extracred}
  %   \caption{Bouncing and pitching motion as a function of time}
  %   \label{fig:bounceAndPitch}
  % \end{figure}

  % \begin{table}[!ht]
  %   \begin{center}
  %     \caption{Calculated values}
  %     \label{tab:calculated}
  %     \begin{tabular}{|c|c|}
  %       \hline
  %     \end{tabular}
  %   \end{center}
  % \end{table}

  % Appendix \onecolumn \textwidth=456pt \paperwidth=577pt \hoffset=-30pt
  % \newpage
  % \newpage
  % \clearpage


  % \pagestyle{headings}
  % \section{Appendix A}
  % \centering Heading \footnotesize

  % \normalsize

  % \csvautolongtable{assets/vibProj.csv}
